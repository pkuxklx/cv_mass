\subsection{图像分割}
% 讲述遇到的困难:轮廓重叠,
\subsubsection{基于最小二乘法的椭圆拟合}
我们使用椭圆来近似猪的俯视图。椭圆可以用二阶多项式方程的解表示:
\begin{equation}
    F_{\bf a}(x,y) = {\bf x^T a} = 0 \text{ 且 } {\bf a^T C  a} > 0  
\end{equation} 
% 花括号{}限定了\bf的作用域
其中
\begin{eqnarray}
    \bf{a} &=& [a,b,c,d,e,f]^T \\
    \bf x &=& [x^2,xy,y^2,x,y,1]^T \\
    {\bf C} &=& \begin{pmatrix}
        \begin{pmatrix}
            0 & 0 & 2 \\
            0 & -1 & 0 \\
            2 & 0 & 0
        \end{pmatrix} & O_{3\times 3} \\
        O_{3\times 3} & O_{3\times 3}
    \end{pmatrix}_{6\times 6}
\end{eqnarray}
为了唯一确定系数,我们规定
\begin{equation}
    {\bf a^T C a} - 1 = 0  
    \label{constraint}
\end{equation}

对于某组轮廓的点集$\{(x_i, y_i)|i=1,\dots, N \}$,记数据矩阵
\begin{equation}
    \bf X = \begin{pmatrix}
        x_1^2 & x_1y_1 & y_1^2 & x_1 & y_1 & 1 \\
        \vdots & \vdots & \vdots & \vdots & \vdots & \vdots \\
        x_i^2 & x_iy_i & y_i^2 & x_i & y_i & 1 \\
        \vdots & \vdots & \vdots & \vdots & \vdots & \vdots \\
        x_N^2 & x_Ny_N & y_N^2 & x_N & y_N & 1 \\
    \end{pmatrix} 
\end{equation}
,则$\bf a$的最小二乘估计为
\begin{equation}
    \hat {\bf a}_{OLS} = \arg \min_{\bf a} \|{\bf X a }\|^2 \ \text{s.t.} \ {\bf a^T C a} - 1 = 0
\end{equation}
这是一个带约束的优化问题,使用拉格朗日乘子法求解
\begin{equation}
    f({\bf a}, \lambda) = \|{\bf Xa}\|^2 + \lambda (1 - {\bf a^T C a}) 
\end{equation}
令$f$的偏导数为0,我们得到方程组(\label{system Eqns})
\begin{equation}
    \left\{
    \begin{aligned}
        {\bf X^T X a} &=& \lambda {\bf Ca} \\
        {\bf a^T C a} &=& 1  
    \end{aligned}
    \right.
    \label{system Eqns} 
\end{equation}
注意到对于(\ref{system Eqns})的解,有$\|\bf{X a}\|^2 = \lambda$。所以多组解中,与$\lambda$的最小非负解对应的$\bf a$即为所求的$\hat {\bf a}_{OLS}$。

\subsubsection{椭圆预筛选准则}
对于某组轮廓的点集$\{(x_i, y_i)|i=1,\dots, N \}$,每次从中随机选择6个点求解椭圆。此后用另一个参数向量$(l_x, l_y, a, b, \theta)^T$来表征求解的椭圆。其中$(l_x,l_fy)$表示椭圆中心;$(a,b)$表示椭圆的长短轴;$\theta$表示椭圆长轴以$x$轴为起点逆时针旋转的角度,且$0^{\circ} \leq \theta < 180^{\circ}$。

显然,只有部分求解的椭圆都很好地代表了某只猪的轮廓,需要进一步筛选“合理”的椭圆。“合理”的椭圆满足以下3个条件:
\begin{enumerate}
    \item 椭圆的面积$18000<S_{ellipse}<30000$;
    \item 长短轴之比$0.2<\frac{b}{a}<0.4$;
    \item 与轮廓的重叠率$\frac{|P_{ellipse} \cap P_{contour}|}{|P_{ellipse}|}>0.8$,其中$P_{ellipse}$表示椭圆内的像素点的集合。
\end{enumerate}
满足以上的条件才会被放入数据集,以用于下一步的聚类。

\subsubsection{基于K-Means聚类分离轮廓重叠的猪}
二值化图像的每一组轮廓可能由多只猪组成。因此我选择用K-Means聚类区别同一轮廓中的各只猪。

使用K-Means前,要先确定给定轮廓内猪的个数$K$,但$K$



% \subsection{ID检测}
% \subsubsection{提取ID标识}
% \subsubsection{基于傅里叶变换的ID匹配}


